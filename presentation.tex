\documentclass{beamer}
\title{An Intro to Functional Programming}
\subtitle{A Better Way to Program if you like Math and Stuff}
\author{Thomas Gebert}
\date{February 6, 2015}

\usefonttheme{serif}
\usepackage{listings}
\usepackage{color}
\definecolor{lightpurple}{rgb}{0.8,0.8,1}

\lstset{
	stepnumber=1,
	numberstyle=\small\color{black},
	basicstyle=\ttfamily,
	keywordstyle=\color{black},
	commentstyle=\color{black},
	stringstyle=\color{black},
	frame=single,
	tabsize=2,
	backgroundcolor=\color{lightpurple}
	}
 % Define listings style

\usetheme{Warsaw}

\begin{document}
  \frame{\titlepage}
  \begin{frame}
  	\frametitle{What is Functional Programming?} 
  	Functional Programming has several definitions. 
	  \begin{itemize}
	    \item Programming in regards to purity, and in particular, how programming relates to mathematics
	    \item Programming with functions.
	  \end{itemize}
    
  \end{frame}
  \begin{frame}
  	\frametitle{But Tom, I already program with functions!}
  	You actually program with subroutine.  There are a few differences: 
  	\begin{itemize}
  		\item  In mathematics, you can always expect the same result when supplying the same argument, following an $ f(x) = y $ pattern
  		\item In most C-style programming languages, you cannot pass functions around like variables.
  	\end{itemize}
  	%More content goes here
  \end{frame}
  \begin{frame}
  	\frametitle{Ok, so what's the big deal about math?}
  	If you utilize math, you can exploit a few useful properties. 
  	\begin{itemize}
  		\item Composition: $$ f\left(g\right) = f \circ g = h $$
	  	\item Partial Application $$g = f\left(5,x\right)$$
  	\end{itemize}
  \end{frame}
\begin{frame}
	\frametitle{Let the computer write the code for you.}
	\begin{itemize}
		\item Writing instructions for the computer to follow is not fun.
		\item Instead, describe what you want the computer to do, not every step of how you want it done. 
	\end{itemize}
\end{frame}
\begin{frame}[fragile]
	\frametitle{Lodash!}
	JavaScript actually allows for both of these properties (and a lot more) using the excellent Lodash library.
	\begin{lstlisting}[basicstyle=\tiny]
	 _.compose
	 var realNameMap = {
	 'pebbles': 'penelope'
	 };
	 
	 var format = function(name) {
	 name = realNameMap[name.toLowerCase()] || name;
	 return name.charAt(0).toUpperCase() + name.slice(1).toLowerCase();
	 };
	 
	 var greet = function(formatted) {
	 return 'Hiya ' + formatted + '!';
	 };
	 
	 var welcome = _.compose(greet, format);
	 welcome('Penelope');
	 // → 'Hiya Penelope!'
	\end{lstlisting}
    
\end{frame}
\begin{frame}[fragile]
	\frametitle{Lodash!}
		JavaScript actually allows for both of these properties (and a lot more) using the excellent Lodash library.
	\begin{lstlisting}
		var greet = function(greeting, name) { 
		  return greeting + ' ' + name; 
		  };
		var hi = _.partial(greet, 'hi');
		hi('fred');
		// → 'hi fred
	\end{lstlisting}
\end{frame}
\begin{frame}
	\begin{center}
		Questions?
	\end{center}
	
\end{frame}
  % etc
\end{document}
